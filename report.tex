\documentclass[10]{article}
\usepackage[pdftex]{graphicx}
\begin{document}
\begin{title}
\huge{\ A REPORT ON WI-FI USAGE IN MAKERERE UNIVERSITY AND REMIDIES TO IMPROVE THE CHALLENGES FACED }
\end{title}
\author{BAMUZIBIRE SOLOMON MUKISA,14/U/3938/PS,214015782}
\section{Table of Contents}
 \subsubsection{Introduction	2}
 \subsubsection{ Background	2}
 \subsubsection{Problem Statement	2}
 \subsubsection{Aim and Objectives}
 \subsubsection{Research Scope	2}
 \subsubsection{METHODOLOGY OF THE STUDY	2}
 \subsubsection{REVIEW OF RELATED LITERATURE	2}
 \subsubsection{INTERNET RESOURCES FOR STUDENTS	2}
 \subsubsection{The Network	2}
 \subsubsection{Campus Challenge	2}
 \subsubsection{References	2}
 
\section{introduction}

The internet is an electronic communications network that connects computer networks and organizational computer facilities around the world. For one to have internet access networking of communicating devices has to occur,  a network is simply a group of two or more computer systems linked together in some way so that they can share data between them. Different types of networks provide different services, and require different things to work properly, taking an example of local Area Network and Wireless Area Network.
A wireless local-area network (LAN) uses radio waves to connect devices such as laptops to the Internet and to your business network and its applications.
Wi-Fi refers to the 802.11b wireless Ethernet standard to support wireless LANs.
A hotspot is a physical location where people may obtain Internet access, typically using Wi-Fi technology, via a wireless local area network (WLAN) using a router connected to an internet service provider.

\subsection{Background}

Wi-Fi is the popular name for the 802.11b wireless Ethernet standard for WLAN. Wireless LANs operate unlicensed use of band within the 2-4 GHz band. The present generation of WLAN supports up to 11 Mbps data transfer speeds less than 100 m from the base station. Most common wireless local area networks have been developed in a distributed way to offer last hundred meters connectivity to a wired network backbone enterprise or campus. Typically, WLAN networks are implemented as part of a private network. Base station equipment owned and operated by the community of end-users within the corporate enterprise, campus, or government network. In most cases, the use of the network is free to end user.
Although each base station can support only connections to a range of one hundred meters, it is possible to ensure a continuous in a wide coverage area using multiple base stations. A number of business companies and universities have developed wireless local area networks, such as continuously. However, the WLAN technology is not designed to support hand-off at high speed associated with the users move between coverage areas of the base stations.
Over the past two years, we have seen the emergence of a number of service providers that offer paid Wi-Fi services in selected local areas, such as hotels, airport lounges and coffee shops also there is a growing movement of so-called'' freenets '' where individuals or organizations to provide subsidized open networks Wi-Fi.
In contrast to the mobile, wireless local area networks has mainly focused on supporting communication data. However, with the growing interest to support real-time traffic such as voice and video over networks IP, it is possible to support voice services WLAN services.
There are four main types of wireless networks:
•	Wireless Local Area Network (LAN): Links two or more devices using a wireless distribution method, providing a connection through access points to the wider Internet.
•	Wireless Metropolitan Area Networks (MAN): Connects several wireless LANs.
•	Wireless Wide Area Network (WAN): Covers large areas such as neighboring towns and cities.
•	Wireless Personal Area Network (PAN): Interconnects devices in a short span, generally within a person’s reach.

\subsection{Aim and Objectives}

a)	Find out the student community awareness on availability of Wi-Fi connectivity.
b)	Better understand the student purpose of using Wi-Fi.
c)	Understand the minute problem face by the student community.
d)	Study the level of ICT know how of Pondicherry university students’.

\subsection{Research Scope}
This research focuses on Hotspots within Makerere University for students to easy Wi-Fi i.e. free Wi-Fi for students within Makerere (Mak air).

\subsection{problem statement}
The problem this document seeks to address is weak internet connectivity and few hotspots within Makerere University.

\subsection{Research }
Purpose of this research is to get information about;
a)	All Wi-Fi hotspots within Makerere University.
b)	The challenges students face to access this Wi-Fi.
c)	Reliability of these hotspots in terms of speed, steady connectivity.

\section{METHODOLOGY OF THE STUDY}

The main goal of this work has been to achieve a complete and objective study of the current wireless networks and their methodological applications, checking their efficiency, environment adaptation, security and cost. 
Another aspect taken in account is the opportunity for teaching to the regional professionals this kind of technology. This is possible because the wireless network is also used as a great research and development platform. Besides, the network offers access to the Internet for students outside the classrooms, such as the library, or the possibility of broadcasting classes by means of the Internet.
So, the goal of the project has been to equip campus with a wireless network which covers practically all the School areas (class-rooms, library, laboratories, offices and common areas).
The data was collected using an online based survey questionnaire. The aim of the questionnaire was to examine the extent of Wi-Fi technology use amongst this Makerere University students.  A sample of students were handed my phone so they could fill in the questionnaire and those that didn’t mind downloading ODK would fill in at their convenience. The students were selected randomly, based on student residing in and off the university campus. The survey was available for 3weeks and 10 members (N= 10) out of 50{uniquely selected} students in the university responded to the invitation by completing the survey. All responses were anonymous. 
The survey comprised of closed ended questionnaire organized around four topics. These were: (1) Knowledge on Wi-Fi, (2) facilities of Wi-Fi (3) knowledge on hotspots, and (4) rating the Wi-Fi facilities performance. For purposes of clarity, the discussion and results will be presented organized around these four topics with the results. 


\includegraphics{may.jpeg}
\includegraphics{zoo.jpeg}
\includegraphics{r4.png}


	
\section{ REVIEW OF RELATED LITERATURE}
The overarching research question guiding the literature review is: how does the literature on Wi-Fi use suggest as the appropriate frameworks and concept for describing and analyzing academic perusal. Internet usage rates per capita are now growing at faster rates in poor countries than in rich countries. According to Johan Lundin over the past 10 years the capacity of a regular, commercial of the shelf laptop has certainly increased. Today, the difference in capacity between a desktop machine and a laptop is rather limited, at least in relation to the way that most of us use our computers. And at the same time, the weight and price have dropped significantly. It is difficult to get reliable and global data on computer sales, but 2005 is argued to be the year when laptops started to sell more than desktop computers (Singer, 2005). Given the widespread adoption of mobile technology among students, combined with increased possibilities for network access (such as Wi-Fi connections in classrooms and across university campuses), and extended battery life, it is understandable that more and more students bring laptops to the classroom. In this sense, the students themselves make computers an important part in their educational activities. Today, teachers expect students to deliver papers written on computers, expecting them to be able to use spell-check, count words, etc. Students are expected to use the digital resources of university libraries, read e-mails, get information about schedules and upload their assignments online. And there has been an increase in the usage of consortia within the campus as the usage increases the cost benefit of electronic resources in the given institution will prove to be effective.
Universities often have numerous issues with their equipment and technologies provided by multiple manufacturers, creating a Frankenstein network. Campus wireless networks are frequently down, equipment management is complicated, different systems do not talk to each other and in some places the network does not exist at all. University campuses are often geographically large areas and require wireless networks that are both powerful and reliable.
Using high-quality, low-cost equipment from manufacturers like Ubiquiti and Mikrotik, Inveneo’s comprehensive solution connects your entire campus using one scalable, easily managed network.


INTERNET RESOURCES FOR STUDENTS
The most effective communication resources, computers and the Internet, are part of our daily life and have become one of the important tools in the education. The Internet helps transfer information between different points therefore this satiation makes the Internet a very powerful information system. People in different age groups and jobs, students and academicians who do scientific research and prepare projects prefer using the Internet because it is the easiest, fastest, and cheapest ways of accessing necessary information (Cloud, 1989). Even though the Internet is a very important and indispensable source for students, the issue of whether the referenced source is trustworthy and/or credible, has been raised. This is because there is no control on any particular piece of information published through the Web, in opposition to the scientific and professional journals published by the scientific institutions, business world and the organizations known to the public. Additionally, other journals and books issued by commercial organizations do not have a control unit including editors and referees. Many of the sites on the Internet enable anybody to submit any kind of information without being controlled, and many of the sites known as reliable are restricted to open access for commercial purposes or security requirements (IP restriction, membership). This limits the accessibility for students and deprives them of these sites. Figure 1 explains the different resources and their accessibility for the students through the Internet. Figure 1. Resource accessibilities for student project TOJET: The Turkish Online Journal of Educational Technology – April 2010, volume 9 Issue 2 Copyright  The Turkish Online Journal of Educational Technology 237 Many of the Internet resources qualified as trustworthy have limited accessibility as shown in the figure. Because of these constraints, information resources used by students are generally untrustworthy or students have been inaccurately forwarded.

The Network
Universities typically work with a local implementation partner to align an Internet service provider and a local technology solutions distributor. A spectrum analysis is performed and Wi-Fi access points are installed to cover the campus and its various public areas.
Users can be validated against a database and authenticated by a RADIUS (Remote Authentication Dial-In User Service) server. Our networks work 100% transparently with various back-end user access systems and IP technologies. For outdoor wireless coverage, outdoor access points are connected to sector antennas to extend their range.
Deployment is easy and quick. Other vendors require multiple controllers – even up to four – and complicated configuration interfaces. In contrast, our network software manages many devices from a single, user-friendly management interface.

Campus Challenge
Universities often have numerous issues with their equipment and technologies provided by multiple manufacturers, creating a Frankenstein network. Campus wireless networks are frequently down, equipment management is complicated, different systems do not talk to each other and in some places the network does not exist at all. University campuses are often geographically large areas and require wireless networks that are both powerful and reliable.
Using high-quality, low-cost equipment from manufacturers like Ubiquiti and Mikrotik, Inveneo’s comprehensive solution connects your entire campus using one scalable, easily managed network. [3]



\section{REMEDIES TO THE ISSUES WITH THE WI-FI AND CCESSING THE WI-FI}
Inundated with complaints from students and university staff, here are five strategies that ensure that will help improve the wireless network at Makerere University;
1. Don't assume your current wireless network provider can handle the job. IT upgrades and replacements frequently involve existing vendors, but in the case of Makerere university’ wireless network, a new provider is needed better than Dicts.
2. Compare new vendors’ side-by-side. Wading through sales pitches and marketing jargon to get through to the real guts of a wireless network isn't easy, so as to select a suitable service provider. It’s important to factor in both the positive and negative reviews when researching solutions from Cisco, Extreme and other providers, and talked to each of them about "how they would provide their best wireless solution, with an emphasis on our three largest buildings, which were experiencing the biggest problems."
3. Look for a system that can handle all devices. While mobile devices are somewhat standardized in the business world, the collegiate environment has become a hotbed for a wide variety of devices, computers, game consoles and other equipment.
To ensure compatibility across the largest number of devices, the college installed an 802.11a system that can also accommodate 802.11g and 802.11n. We need to have dual-band 802.11n, which is the newest standard and offers the most throughput available. 
4. Get users involved in the selection and testing processes. During the upgrade, Rush communicated regularly with students and faculty about the process, asking for their input and feedback along the way. [1]

\section{References}


[1]	M. University, "Makerere University," [Online]. Available: http://www.mak.ac.ug. [Accessed 13 May 2017].
[2] 	E. P. H. society, "Responding to Wi-fi Safety concerns in Schools," september 2014. [Online]. Available: http://www.doh.wa.go. [Accessed 15 May 2017].
[3] 	I. C. WiFi. [Online]. Available: http://http://www.inveneo.org.
[4] 	B. McCrea, "5-ways-to-build-a-better-wireless-network," 04 April 2011. [Online]. Available: https://campustechnology.com/articles/2011/04/14/5-ways-to-build-a-better-wireless-network.aspx. [Accessed 15 may 2017].

\end{document}































