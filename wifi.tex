\documentclass[10]{article}
\begin{document}
\title{concept paper}

\maketitle
\author{BAMUZIBIRE SOLOMON MUKISA,14/U/3938/PS,214015782}
\section{Introduction}
The internet is an electronic communications network that connects computer networks and organizational computer facilities around the world. For one to have internet access networking of communicating devices has to occur,  a network is simply a group of two or more computer systems linked together in some way so that they can share data between them. Different types of networks provide different services, and require different things to work properly, taking an example of local Area Network And Wireless Area Network.
A wireless local-area network (LAN) uses radio waves to connect devices such as laptops to the Internet and to your business network and its applications.
Wi-Fi refers to the 802.11b wireless Ethernet standard to support wireless LANs.
A hotspot is a physical location where people may obtain Internet access, typically using Wi-Fi technology, via a wireless local area network (WLAN) using a router connected to an internet service provider.

\section{Background}
1.1	Background to the problem.
Wi-Fi is the popular name for the 802.11b wireless Ethernet standard for WLAN. Wireless LANs operate unlicensed use of band within the 2-4 GHz band. The present generation of WLAN supports up to 11 Mbps data transfer speeds less than 100 m from the base station. Most common wireless local area networks have been developed in a distributed way to offer last hundred meters connectivity to a wired network backbone enterprise or campus. Typically, WLAN networks are implemented as part of a private network. Base station equipment owned and operated by the community of end-users within the corporate enterprise, campus, or government network. In most cases, the use of the network is free to end user.
Although each base station can support only connections to a range of one hundred meters, it is possible to ensure a continuous in a wide coverage area using multiple base stations. A number of business companies and universities have developed wireless local area networks, such as continuously. However, the WLAN technology is not designed to support hand-off at high speed associated with the users move between coverage areas of the base stations.
Over the past two years, we have seen the emergence of a number of service providers that offer paid Wi-Fi services in selected local areas, such as hotels, airport lounges and coffee shops also there is a growsing movement of so-called'' freenets '' where individuals or organizations to provide subsidized open networks Wi-Fi.
In contrast to the mobile, wireless local area networks has mainly focused on supporting communication data. However, with the growing interest to support real-time traffic such as voice and video over networks IP, it is possible to support voice services WLAN services.
There are four main types of wireless networks:
•	Wireless Local Area Network (LAN): Links two or more devices using a wireless distribution method, providing a connection through access points to the wider Internet.
•	Wireless Metropolitan Area Networks (MAN): Connects several wireless LANs.
•	Wireless Wide Area Network (WAN): Covers large areas such as neighboring towns and cities.
•	Wireless Personal Area Network (PAN): Interconnects devices in a short span, generally within a person’s reach.

\section{Aim and Objectives}

a)	Find out the student community awareness on availability of Wi-Fi connectivity.
b)	Better understand the student purpose of using Wi-Fi.
c)	Understand the minute problem face by the student community.
d)	Study the level of ICT know how of Pondicherry university students’.

\section{Research Scope}
This research focuses on Hotspots within Makerere University for students to easy Wi-Fi i.e. free Wi-Fi for students within Makerere (Mak air).

\section{problem statement}
The problem this document seeks to address is weak internet connectivity and few hotspots within Makerere University.

\section{Research Significance}
Purpose of this research is to get information about;
a)	All Wi-Fi hotspots within Makerere University.
b)	The challenges students face to access this Wi-Fi.
c)	Reliability of these hotspots in terms of speed, steady connectivity.

\end{document}
